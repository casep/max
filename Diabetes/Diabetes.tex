%++++++++++++++++++++++++++++++++++++++++
% Don't modify this section unless you know what you're doing!
\documentclass[letterpaper,12pt]{article}
\usepackage{tabularx} % extra features for tabular environment
\usepackage{amsmath}  % improve math presentation

\usepackage{graphicx}
\graphicspath{{images/}}
\usepackage[section]{placeins}

\usepackage[margin=1in,letterpaper]{geometry} % decreases margins
\usepackage{cite} % takes care of citations
\usepackage[final]{hyperref} % adds hyper links inside the generated pdf file
\hypersetup{
	colorlinks=true,       % false: boxed links; true: colored links
	linkcolor=blue,        % color of internal links
	citecolor=blue,        % color of links to bibliography
	filecolor=magenta,     % color of file links
	urlcolor=blue         
}
%++++++++++++++++++++++++++++++++++++++++
\usepackage[UKenglish]{babel}% http://ctan.org/pkg/babel


\begin{document}

\title{What to expect when diagnosed with Type 2 Diabetes}
\author{Max Sepulveda}
\date{\today}
\maketitle

\begin{abstract}
The following is an information sheet for someone who has been newly diagnosed with the Type 2 Diabetes.

\end{abstract}

\section{Type 2 Diabetes}

As you know, you have been diagnosed with Type 2 Diabetes, but don't be scared, it may change the things you can eat, but if treated responsibly it is not too dangerous.\\

I will now proceed to explain what diabetes is, why it happens and how it can be managed.\\

Type 2 Diabetes is a common condition that causes the level of sugar (glucose) in the blood to become too high.\\

why does this happen you may ask and this is because the insulin which your pancreas creates doesn't work or your pancreas doesn't create enough insulin to keep up with the glucose that the body is freeing from food that you consume.\\

The insulin that does not work which your pancreas creates avoids the glucose from going into the bloodstream and this makes it  accumulate thus the glucose level rises.\\

Some symptoms of type 2 diabetes are:
\begin{itemize}
 \item Feeling tired.
 \item Having to urinate a lot.
 \item Wounds healing very slowly.
 \item Being very thirsty.
 \item Getting infections like thrush as well as having blurry vision.
\end{itemize}

\section{How can you manage this?}
There are many ways to manage type 2 diabetes some of which are:
\begin{itemize}
 \item Eating healthy (eating food rich on fibre and healthy carbohydrates).
 \item Being more active, doing sports and going for walks.
 \item and losing weight.
\end{itemize}

\section{What are low GI Foods?}
The GI (glycemic index) is a number from 0 to 100 assigned to a food, with pure glucose arbitrarily given the value of 100, which represents the relative rise in the blood glucose level two hours after consuming that food.\\

It is recommended to eat low GI food because they are more slowly digested, which causes a slower rise of glucose in the blood which your pancreas can manage.
Some GI foods are:
\begin{itemize}
 \item $100\%$ stone-ground whole wheat or pumpernickel bread.
 \item Oatmeal (rolled or steel-cut), oat bran, muesli.
 \item Pasta, converted rice, barley, bulgur.
 \item Sweet potato, corn, yam, lima/butter beans, peas, legumes and lentils.
 \item Most fruits, non-starchy vegetables and carrots.
\end{itemize}

\nocite{whatis}
\nocite{type2}

\section{Bibliography}

\begin{thebibliography}{99}

\bibitem{whatis}
What is Type 2 diabetes?
    \href{https://www.diabetes.org.uk/diabetes-the-basics/what-is-type-2-diabetes}{https://www.diabetes.org.uk/diabetes-the-basics/what-is-type-2-diabetes}

\bibitem{type2}
Type 2 diabetes
    \href{https://www.nhsinform.scot/illnesses-and-conditions/diabetes/type-2-diabetes}{https://www.nhsinform.scot/illnesses-and-conditions/diabetes/type-2-diabetes}


\end{thebibliography}


\end{document}
